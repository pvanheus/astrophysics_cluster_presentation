%% LaTeX Beamer presentation template (requires beamer package)
%% see http://bitbucket.org/rivanvx/beamer/wiki/Home
%% idea contributed by H. Turgut Uyar
%% template based on a template by Till Tantau
%% this template is still evolving - it might differ in future releases!

\documentclass[handout]{beamer}

\mode<presentation>
{
\usetheme{Warsaw}

\setbeamercovered{transparent}
}

\usepackage[english]{babel}
\usepackage[utf8x]{inputenc}
\usepackage{pgf}
\usepackage{setspace}
\usepackage{listings}
\usepackage{hyperref}
\usepackage[round]{natbib}
\renewcommand{\cite}[1]{{\small\citep{#1}}} 
\def\newblock{\hskip .11em plus .33em minus .07em}
\usepackage{numprint}

% font definitions, try \usepackage{ae} instead of the following
% three lines if you don't like this look
\usepackage{mathptmx}
\usepackage[scaled=.90]{helvet}
\usepackage{courier}


\usepackage[T1]{fontenc}

\pgfdeclareimage[height=0.15\textheight]{UWC-logo}{images/UWClogo.png}
\logo{\href{http://www.uwc.ac.za}{\pgfuseimage{UWC-logo}}}

% \setbeamertemplate{sidebar left}{%
%    \vfill%
%    \rlap{\hskip0.1cm%
%          \href{http://www.uwc.ac.za}%
%          {\pgfuseimage{UWC-logo}}}%
%    \vskip2pt%
%    \llap{\usebeamertemplate***{navigation symbols}\hskip0.1cm}%
%    \vskip2pt%
% } 

\title{UWC Astrophysics Cluster}

\subtitle{Low Cost High Performance Computing}

% - Use the \inst{?} command only if the authors have different
%   affiliation.
%\author{F.~Author\inst{1} \and S.~Another\inst{2}}
\author{\texorpdfstring{Peter~van~Heusden\newline
~pvh@sanbi.ac.za}{Peter~van~Heusden}}


% - Use the \inst command only if there are several affiliations.
% - Keep it simple, no one is interested in your street address.
\institute[UWC]
{
Information and Communication Services\\
University of the Western Cape\\
Bellville, South Africa\\
\insertlogo}

\date{September 2014}

%\beamerdefaultoverlayspecification{<+->}


% This is only inserted into the PDF information catalog. Can be left
% out.
\subject{Talks}



% If you have a file called "university-logo-filename.xxx", where xxx
% is a graphic format that can be processed by latex or pdflatex,
% resp., then you can add a logo as follows:

% \pgfdeclareimage[height=0.5cm]{university-logo}{university-logo-filename}
% \logo{\pgfuseimage{university-logo}}



% Delete this, if you do not want the table of contents to pop up at
% the beginning of each subsection:
% \AtBeginSubsection[]
% {
% \begin{frame}<beamer>
% \frametitle{Outline}
% \tableofcontents[currentsection,currentsubsection]
% \end{frame}
% }

% If you wish to uncover everything in a step-wise fashion, uncomment
% the following command:

%\beamerdefaultoverlayspecification{<+->}
\setbeamertemplate{bibliography item}[text]

\begin{document}

\begin{frame}
\logo{}
\titlepage
\end{frame}

\section{Executive Summary}
\begin{frame}
\frametitle{UWC Astrophysics Cluster}
\begin{itemize}
\item 23 worker nodes, 2 master nodes in two clusters
\item Built from Supermicro servers, AMD Opteron processors
\pause
\item Timon cluster: 
\begin{itemize}
\item 4 worker nodes, 192 cores
\item 1024 GB RAM
\item 39 TB filesystem distributed over worker nodes (CephFS)
\end{itemize}
\pause
\item{Pumbaa cluster:}
\begin{itemize}
\item 19 worker nodes, 912 cores
\item 4864 GB RAM
\item 133 /data distributed filesystem (GlusterFS)
\end{itemize}
\end{itemize}
\end{frame}

\pgfdeclareimage[height=0.8\textheight]{Cluster-Diagram}{images/cluster_diagram.png}
\begin{frame}[fragile]
\frametitle{Astrophysics Cluster Diagram}
\begin{columns}
\column{1.2\textwidth}{
  % \hspace{-1cm}
  \pgfuseimage{Cluster-Diagram}
}
\end{columns}
\end{frame}

\begin{frame}
\frametitle{Networking}
\begin{itemize}
\item Each cluster has a 10 Gb Ethernet and 1 Gb Ethernet switch
\item 1 GbE provides management layer
\begin{itemize}
\item Used for initial software installation, authentication and cluster management
\end{itemize}
\item 10 GbE used for parallel computing and distributed filesystem
\begin{itemize}
\item Each cluster node (and master) has dual 10GbE twisted pair connections
\item Allows for independent data paths for MPI (parallel computing) and filesystem (storage)
\end{itemize}
\item Two clusters are connected by VLAN operating on ICS network core switch
\begin{itemize}
\item No high throughput or low latency traffic between clusters
\end{itemize}
\end{itemize}
\end{frame}

\section{Cluster Deployment and Management}

\pgfdeclareimage[width=\textwidth]{Cluster-Management}{images/cluster_management.png}
\begin{frame}
\frametitle{Cluster Deployment and Management}
\pgfuseimage{Cluster-Management}
\end{frame}

\end{document}
