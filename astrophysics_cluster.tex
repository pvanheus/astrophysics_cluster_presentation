%% LaTeX Beamer presentation template (requires beamer package)
%% see http://bitbucket.org/rivanvx/beamer/wiki/Home
%% idea contributed by H. Turgut Uyar
%% template based on a template by Till Tantau
%% this template is still evolving - it might differ in future releases!

\documentclass[handout]{beamer}

\mode<presentation>
{
\usetheme{Warsaw}

\setbeamercovered{transparent}
}

\usepackage[english]{babel}
\usepackage[utf8x]{inputenc}
\usepackage{pgf}
\usepackage{setspace}
\usepackage{listings}
\usepackage{hyperref}
\usepackage[round]{natbib}
\renewcommand{\cite}[1]{{\small\citep{#1}}} 
\def\newblock{\hskip .11em plus .33em minus .07em}
\usepackage{numprint}

% font definitions, try \usepackage{ae} instead of the following
% three lines if you don't like this look
\usepackage{mathptmx}
\usepackage[scaled=.90]{helvet}
\usepackage{courier}


\usepackage[T1]{fontenc}

\pgfdeclareimage[height=0.15\textheight]{UWC-logo}{images/UWClogo.png}
\logo{\href{http://www.uwc.ac.za}{\pgfuseimage{UWC-logo}}}

% \setbeamertemplate{sidebar left}{%
%    \vfill%
%    \rlap{\hskip0.1cm%
%          \href{http://www.uwc.ac.za}%
%          {\pgfuseimage{UWC-logo}}}%
%    \vskip2pt%
%    \llap{\usebeamertemplate***{navigation symbols}\hskip0.1cm}%
%    \vskip2pt%
% } 

\title{UWC Astrophysics Cluster}

\subtitle{Low Cost High Performance Computing}

% - Use the \inst{?} command only if the authors have different
%   affiliation.
%\author{F.~Author\inst{1} \and S.~Another\inst{2}}
\author{\texorpdfstring{Peter~van~Heusden\newline
~pvh@sanbi.ac.za}{Peter~van~Heusden}}


% - Use the \inst command only if there are several affiliations.
% - Keep it simple, no one is interested in your street address.
\institute[UWC]
{
Information and Communication Services\\
University of the Western Cape\\
Bellville, South Africa\\
\insertlogo}

\date{September 2014}

%\beamerdefaultoverlayspecification{<+->}


% This is only inserted into the PDF information catalog. Can be left
% out.
\subject{Talks}



% If you have a file called "university-logo-filename.xxx", where xxx
% is a graphic format that can be processed by latex or pdflatex,
% resp., then you can add a logo as follows:

% \pgfdeclareimage[height=0.5cm]{university-logo}{university-logo-filename}
% \logo{\pgfuseimage{university-logo}}



% Delete this, if you do not want the table of contents to pop up at
% the beginning of each subsection:
% \AtBeginSubsection[]
% {
% \begin{frame}<beamer>
% \frametitle{Outline}
% \tableofcontents[currentsection,currentsubsection]
% \end{frame}
% }

% If you wish to uncover everything in a step-wise fashion, uncomment
% the following command:

%\beamerdefaultoverlayspecification{<+->}
\setbeamertemplate{bibliography item}[text]

\begin{document}

\begin{frame}
\logo{}
\titlepage
\end{frame}

\begin{frame}
\frametitle{UWC Astrophysics Cluster}
\begin{itemize}
\item 23 worker nodes, 2 master nodes in two clusters
\item Built from Supermicro servers, AMD Opteron processors
\pause
\item Timon cluster: 
\begin{itemize}
\item 4 worker nodes, 192 cores
\item 1024 GB RAM
\item 39 TB filesystem distributed over worker nodes (CephFS)
\end{itemize}
\pause
\item{Pumbaa cluster:}
\begin{itemize}
\item 19 worker nodes, 912 cores
\item 4864 GB RAM
\item 133 /data distributed filesystem (GlusterFS)
\end{itemize}
\end{itemize}
\end{frame}

\pgfdeclareimage[height=0.8\textheight]{Cluster-Diagram}{images/cluster_diagram.png}
\begin{frame}[fragile]
\frametitle{Astrophysics Cluster Diagram}
\begin{columns}
\column{1.2\textwidth}{
  % \hspace{-1cm}
  \pgfuseimage{Cluster-Diagram}
}
\end{columns}
\end{frame}

\begin{frame}
\frametitle{Networking}
\begin{itemize}
\item Each cluster has a 10 Gb Ethernet and 1 Gb Ethernet switch
\item 1 GbE provides management layer
\begin{itemize}
\item Used for initial software installation, authentication and cluster management
\end{itemize}
\item 10 GbE used for parallel computing and distributed filesystem
\begin{itemize}
\item Each cluster node (and master) has dual 10GbE twisted pair connections
\item Allows for independent data paths for MPI (parallel computing) and filesystem (storage)
\end{itemize}
\item Two clusters are connected by VLAN operating on ICS network core switch
\begin{itemize}
\item No high throughput or low latency traffic between clusters
\end{itemize}
\end{itemize}
\end{frame}

\begin{frame}
\frametitle{One, two, many HPCs}
\begin{itemize}
  \item Before 1990, HPC meant hugely expensive computers built on national or
  regional scale
  \item Institute-level computing in 1990s was dominated by individual SMP
  machines
  \item Mid-to-late 90s saw birth of the commodity cluster (Beowulf)
\end{itemize}
\end{frame}

\begin{frame}
\frametitle{One, two, many HPCs (cont)}
\begin{itemize}
  \item Now: HPC everywhere, but skills hard to find
  \item The few courses that exist focus on HPC programming (parallel
  applications, MPI, etc)
  \item HPC exists alongside ``cyberinfrastructure'' and ``cloud'': different
\  environments but all based on complex and often poorly documented software
\end{itemize}
\begin{quote}
``with limited resources and expertise, even simple data discovery, collection,
analysis, management, and sharing tasks are difficult for small teams'' (Ian
Foster, Computation Institute)
\end{quote}
\end{frame}

\begin{frame}
\frametitle{How to become a (better) HPC admin}
\begin{itemize}
  \item \uncover<1->{Most HPC admin knowledge is passed on peer-to-peer}
  \item \uncover<1->{Admins rely on the ``social media'' of HPC: Mailing lists,
  technical blogs, IRC}
  \item \uncover<1->{Networking happens sporadically} 
  \begin{itemize}
    \item \uncover<2->{- HPC admins don't get out much}
  \end{itemize} 
\end{itemize}
\end{frame}

\begin{frame}
\frametitle{The HPC Forum plan}
\begin{itemize}
  \item First face to face seminar day: mid-February 2014
  \item Minimal wiki:
  \href{http://hpcforum.sanbi.ac.za}{http://hpcforum.sanbi.ac.za}
  \begin{itemize}
    \item Will soon move to
    \href{http://hpcforum.org.za}{http://hpcforum.org.za}
  \end{itemize}
  \item Grow (mailing list) membership
  \item Keep producing (and linking to) tech blogs
\end{itemize}
\end{frame}

\begin{frame}
\frametitle{HPC research and teaching}
\begin{itemize}
    \item HPC admin has historically been seen as an operations issue
    \item Academic interest has been limited and curricula non-existent
    \pause
    \item HPC Forum could:
    \begin{itemize}
      \item Collect and curate teaching materials on HPC admin
      \item Provide space for learning and experimentation in HPC
      \item Increase the profile of HPC as an area of research and learning
      \item Create a collective experimental community for HPC technology
      \item Contribute ``infrastructure code'' to a shared repository (e.g.
      SAGridOps on Github)
    \end{itemize}
\end{itemize}
\end{frame}

\begin{frame}
\frametitle{Related initiatives}
\begin{itemize}
  \item Africa-Arabia Regional Operations Centre
  \pause
  \item SADC HPC Forum : human capital development in HPC
\end{itemize}
\end{frame}

\begin{frame}[plain,c]
\begin{center}
\Huge{Over to you.}
\end{center}
\end{frame}

% \begin{frame}
% \frametitle{Outline}
% \tableofcontents
% % You might wish to add the option [pausesections]
% \end{frame}


\end{document}
